\documentclass[../main.tex]{subfiles}
 
\begin{document}
 
\begin{figure}[bh]
\centering
 
\end{figure}

\subsection{Cel pracy}

  W pracy przedstawione zostanie wykorzystanie rozszerzonej wersji algorytmu "A*" do planowania ruchu pojazdów na skrzyżowaniach dróg wielopasmowych. Użycie algorytmu ma koordynować ruch drogowy uwzględniając przy tym bezpieczeństwo pojazdów. Rozwiązanie ma także uwzględniać potencjalne błędy kierowców polegające na nie dostosowaniu się do przesłanego im planu.
  Ruch na skrzyżowaniach będzie koordynowany następująco. Każdy z kierowców dostaje na bieżąco plan, który mówi mu, jak ma on się poruszać na drodze. Plan zawiera informacje z jaką prędkością kierowca powinien się poruszać oraz kiedy powinien zwolnić lub przyspieszyć. Rozwiązanie uwzględnia błędy kierowców. W razie, gdy kierowca nie zastosuje się do planu - błąd jest wykrywany, następnie liczone są nowe plany oraz wysyłane do kierowców. Bezpieczeństwo przed kolizją zapewniane jest poprzez zachowanie odpowiedniej, określonej odległości między autami.
  Pokazane w pracy podejście powinno usprawnić ruch drogowy na skrzyżowaniach w porównaniu do rozwiązania stosowanego najczęściej, czyli użycie świateł drogowych. Światła drogowe zapewniają bezpieczeństwo poprzez światło pomarańczowe. Światło pomarańczowe na skrzyżowaniu, to moment, w którym auta stoją, przez co tracą na czasie. Co za tym idzie ruch na skrzyżowaniu jest mniej sprawny.

\subsection{A* w planowaniu ruchu drogowego}

  Algorytm A* to algorytm heurystyczny znajdowania najkrótszej ścieżki w grafie ważonym z dowolnego wierzchołka do wierzchołka spełniającego określony warunek zwany testem celu.  
\end{document}
