\documentclass[../main.tex]{subfiles}
 
\begin{document}
 
\begin{figure}[bh]
\centering
 
\end{figure}

\subsection{Cel pracy}

W pracy przedstawione zostanie wykorzystanie rozszerzonej wersji algorytmu "A*" do planowania ruchu pojazdów na skrzyżowaniach dróg wielopasmowych. Celem pracy jest pokazanie alternatywy dla koordynacji ruchu pojazdów przy użyciu świateł drogowych. Użycie algorytmu ma koordynować ruch drogowy uwzględniając przy tym bezpieczeństwo pojazdów. Rozwiązanie ma także uwzględniać potencjalne błędy kierowców polegające na nie dostosowaniu się do przesłanego im planu.

\subsection{Motywacja}

Koordynacja ruchu za pomocą rozszerzonego algorytmu "A*" powinna działać szybciej niż koordynacja ruchu za pomocą świateł drogowych. Wynika to z faktu, że bezpieczeństwo na skrzyżowaniach, gdzie ruch koordynowany jest za pomocą świateł drogowych jest zapewnione poprzez czekanie na świetle żółtym. Porównując to z rozwiązaniem, gdzie każdy kierowca otrzymuje plan, zgodnie z którym powinien się poruszać - czas przeznaczony na bezpieczeństwo powinien być krótszy. Wynikiem tego, ruch na skrzyżowaniach powinien przebiegać sprawniej.


\subsection{Finalne rozwiązanie}

Rozwiązanie będzie uwzględniać błędy kierowców, polegające na nie dostosowaniu się do planu. W takim wypadku, pomyłka zostanie wykryta i kierowca dostanie nowy plan. Problemem jest złożoność obliczeniowa - wyliczenie takiego planu zajmuje dużo czasu. Czas ten rośnie wykładniczo do ilości aut na skrzyżowaniach. W razie pomyłki kierowcy plan powinien być dostarczony natychmiastowo. Z tego powodu rozważone będzie wykonwanie planów przewidujących pomyłki kierowców co wiąże się z potrzebą posiadania dużych zasobów obliczeniowych, co jest możliwe w przyszłości.

\end{document}
