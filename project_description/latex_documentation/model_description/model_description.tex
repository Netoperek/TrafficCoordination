\documentclass[../main.tex]{subfiles}
 
\begin{document}
 
\begin{figure}[bh]
\centering
 
\end{figure}

\subsection{Definicja problemu}

Podany jest zbiór dróg oraz informacja o tym, które drogi się ze sobą przecinają. Drogi są podzielone na odcinki.
Droga następujące parametry:

\begin{enumerate}  
\item Unikalny numer
\item Rozmiar - liczba odcinków
\item Informacja o tym, z którymi drogami się przecina oraz w którym odcinku
\end{enumerate}


Podany jest zbiór aut, które mają następujące parametry:

\begin{enumerate}  
\item Unikalny numer auta
\item Numer drogi na której się znajduje
\item Numer odcinku drogi na której się znajduje
\item Aktualna prędkość. Prędkość to liczba odcinków drogi przejechana w jednym odcinku czasowym.
\end{enumerate}

Samochód może zwalniać lub przyspieszać o wartości 1 i 2.

\newpage
Wizualizacja modelu:
\newline

\includegraphics[width=0.9\textwidth]{./model_description/1.eps}
\newline
\newline

Drogi są przedstawione w postaci punktów. Punkty w jednej drodze mają taki sam kolor. Pogrubione koła to samochody. Samochody jadące po jednej drodze mają ten sam kolor.

\end{document}
