\documentclass[../main.tex]{subfiles}
 
\begin{document}
 
\begin{figure}[bh]
\centering
 
\end{figure}

\subsection{Opis algorytmu A*}

Algorytm A* to algorytm heurystyczny, znajdujący najkrótszą ścieżkę w grafie ważonym z dowolnego wierzchołka do podanego wierzchołka, który spełnia okerślony warunek. Algorytm jest zupełny i optymalny, w tym sensie, że znajduje ścieżkę, jeśli tylko taka istnieje, i przy tym jest to ścieżka najkrótsza.


Algorytm A* od wierzchołka początkowego tworzy ścieżkę, za każdym razem wybierając wierzchołek x z dostępnych w danym kroku niezbadanych wierzchołków tak, by minimalizować funkcję f(x) zdefiniowaną:

\[ f(x) = g(x) + h(x) \]

gdzie:
\newline
g(x) – droga pomiędzy wierzchołkiem początkowym a x. Dokładniej: suma wag krawędzi, które należą już do ścieżki plus waga krawędzi łączącej aktualny węzeł z x
\newline
h(x) – przewidywana przez heurystykę droga od x do wierzchołka docelowego
\newline

W każdym kroku algorytm dołącza do ścieżki wierzchołek o najniższym współczynniku f. Kończy w momencie natrafienia na wierzchołek będący wierzchołkiem docelowym.

\subsection{Opis rozszerzenia algorytmu A*}

Rozszerzenie algorytmu A* polega na tym, że wierzchołek przedstawia stan, natomiast krawędź przedstawia przejście z jednego stanu do drugiego. W tym konkretnym problemie wierzchołkiem będzie położenie wszystkich aut na skrzyżowaniach wraz z ich aktualnymi prędkościami. 
\newline
\newline
Stan końcowy to stan, w którym wszystki auta przejadą wszystkie napotkane przez siebie skrzyżowania.


\subsection{Funkcja heurystki}




\end{document}
