\mychapter{3}{Teza} \label{chap:thesis}


\section{Wprowadzenie}

W pracy rozwiązywany jest problem planowania ruchu na skrzyżowaniach dróg wielopasmowych. Rozwiązanie w porównaniu do wcześniej opisanych podejść, wyróżnia się użyciem algorytmu A* do dokładnego zaplanowania ruchu dla każdego z pojazdów wraz z unikaniem kolizji. Większość przedstawionych rozwiązań opiera się o światła drogowe, co wiąże się z opóźnieniami spowodowanymi oczekiwaniem na świetle pomarańczowym, a także opóźnieniami spowodowanymi na oczekiwaniu na świetle czerwonym, kiedy na drodze poprzecznej nie ma żadnych pojazdów. Implementacja algorytmu A* jest generyczna i opiera się na stanach. Algorytm przyjmuje obiekt klasy implementującej metodę zwracającą stany sąsiednie dla stanu aktualnego. Do algorytmu przekazywane są także warunki końcowe. Algorytm przeszukuje stany korzystając z funkcji heurystyki, aż osiągnięte zostaną warunki końcowe - nie przeszukiwane są wszystkie możliwe stany. Algorytm kończy obliczenia, kiedy znajdzie stany spełniające określone warunki. Algorytm A*, moduł graficzne oraz pozostałe elementy systemu zostały napisane w języku Ruby.

\section{Planowanie ruchu przy użyciu modyfikacji A*}

Można postawić tezę: zastosowanie metody planującej ruch drogowy przy użyciu algorytmu A* pozwala na zaplanowanie bezkolizyjnego ruchu drogowego dla pojazdów na skrzyżowaniu, co pozwala na osiągnięcie lepszej przepustowości niż w przypadku świateł sekwencyjnych wraz z optymalizacjami. Przepustowość rozumiemy jako ilość samochodów opuszczających skrzyżowanie w jednostce czasu. Rozwiązanie jest uruchamiane na jednym rdzeniu procesora. Korzystając z większej ilości rdzeniów, metoda może policzyć plany dla wielu wariantów. Plany wielowariantowe znajdują zastosowanie w przypadku pomyłki pojazdów, polegającej na niezastosowanie się do wyliczonego planu.
\newline
\indent
Rozwiązanie przyjmuje położenie samochodów na sieci skrzyżowań oraz informacje na temat dróg jako dane wejściowe. Wynikiem jest dokładny plan poruszania się dla każdego z pojazdów. W planie znajdują się informacje o tym z jaką prędkością pojazd powinien się poruszyć, kiedy zwolnić lub przyspieszyć, aby omijać kolizje.
\newline
\indent
W metodzie, do planowania wykorzystany został algorytm A*. Jest on oparty o działanie na stanach, gdzie pojedynczym stanem jest rozmieszczenie pojazdów na sieci skrzyżowań wraz z ich prędkościami. Stanem sąsiednim dla aktualnego stanu, jest zmiana pozycji pojazdów na sieci skrzyżowań wraz z ich prędkościami. Implementacja algorytmu A* jest generyczna - algorytm przyjmuje dowolny obiekt klasy implementujący metodę 'neighbours', która zwraca stany sąsiednie dla aktualnego stanu. Algorytm działa w sposób dynamiczny. W przypadku działania algorytmu A* na grafie, z góry znane są krawędzie oraz wierzchołki. W opracowanym rozwiązaniu algorytm dynamicznie tworzy sąsiednie stany.
\newline
\indent
Do algorytmu przekazywana jest funkcja heurystyki, a także warunek końcowy. Dla koordynacji ruchu, warunkiem końcowym jest opuszczenie przez wszystkie pojazdy skrzyżowania. Funkcja heurystyki operuje na danych stanu - każdemu stanowi przypisuje wartość. W przedstawionym rozwiązaniu funkcja heurystyki zwraca najmniejsze wartości dla stanów, w których samochody maksymalnie przyspieszają.
\newline
\newline
\indent
Do systemu wprowadza się także następujące dane:
\begin{itemize}
\item maksymalną prędkość jaką pojazd może osiągnąć
\item wartości przyspieszeń pojazdu
\item parametr bezpieczeństwa, czyli odległość utrzymywana pomiędzy pojazdami
\end{itemize}
System posiadając stan początkowy i powyższe dane, tworzy wszystkie możliwe stany pochodne. Następnie korzystając z funkcji heurystyki algorytm wybiera stany, które prowadzą do najszybszego opuszczenia przez samochody skrzyżowania.
\newline
\indent
Rozwiązanie bierze pod uwagę unikanie kolizji pojazdów na skrzyżowaniu oraz pojazdów poruszających się tym samym pasem. Unikanie kolizji polega na usuwaniu stanów sąsiednich, które powodują kolizję - metoda 'neighbours' zwraca jedynie stany bezkolizyjne.
\newline
\indent
Do systemu można przekazać każdy rodzaj skrzyżowania jako listę dróg wraz z ich rozmiarami oraz informację o przecięciach z innymi drogami. Rozwiązanie oparte jest o przestrzeń dyskretną. Drogi mają konkretny rozmiar. Do systemu przekazuje się także listę samochodów, która zawiera informacje o położeniu samochodów na drogach oraz informację o ich prędkościach. System opiera się o działanie na krokach czasowych. Prędkość samochodów wyrażana jest w liczbie odcinków drogi na jeden krok czasowy.
\newline
\indent
W celu wizualizacji stworzony został także moduł graficzny, który jest w stanie zaprezentować każde podane skrzyżowanie. Przedstawia on także w formie zdjęć, zaplanowany ruch samochodów. Każde zdjęcie odpowiada następnemu krokowi czasowemu. Na ostatnim zdjęciu skrzyżowanie jest opuszczone przez wszystkie samochody.
\newline
\indent
Wadą metody jest złożoność obliczeniowa, która jest zależna od ilości samochodów na skrzyżowaniach. Czas trwania programu zależy wykładniczo od ilości pojazdów na skrzyżowaniu. Dla 14 pojazdów znajdujących się na skrzyżowaniu 8 dróg obliczenia trwają do kilku godzin, co jest znaczącym limitem dla rozwiązania.
