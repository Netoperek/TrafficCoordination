\chapter{Teza} \label{chap:thesis}

\section{Cel}

W opracowanym rozwiązaniu wykonany został system plaujący ruch drogowy na skrzyżowaniach dróg wielopasmowych przy użyciu zmodyfikowanego algorytmu A*. Wynikiem rozwiązania jest szczegółowy opis, który dokładnie planuje ruch każdego z samochodów w celu najszybszego i bezpiecznego opuszczenia skrzyżowania. System zapewnia unikanie kolizji oraz optymalny dojazd do celu pojazdów na zadanej sieci skrzyżowań.

\section{Planowanie ruchu przy użyciu modyfikacji A*}

Rozwiązanie, mając podane położenie samochodów na sieci skrzyżowań liczy plan dla każdego z pojazdów. W planie znajdują się informacje jak szybko pojazd powinien jechać, kiedy powinien zwolnić lub przyspieszyć aby ominąć kolizję.
\newline
\newline
Algorytm A* jest oparty o działanie na stanach. Pojedyńczym stanem jest rozmieszczenie pojazdów na sieci skrzyżowań wraz z ich prędkościami.
\newline
\newline
Do systemu wprowadza się także następujące dane:
\begin{itemize}
\item maksymalną prędkość jaką pojazd może osiągnąć
\item wartośći przyspieszeń pojazdu
\item parametr bezpieczeństwa, czyli odległość utrzymywana pomiędzy pojazdami
\end{itemize}
System posiadając stan początkowy i powyższe dane, tworzy wszystkie możliwe stany pochodne. Następnie korzystając z funkcji heurystyki wybiera najkrótszą ścieżkę stanów.
\newline
\newline
Opracowana została modyfikacja algorytmu A*. Algorytm został zaimplementowany w języku Ruby i jest on generyczny. Algorytm jako stan przyjmuje dowolny obiekt klasy implementujący metodę 'neighbours', która zwraca stany sąsiednie dla aktualnego stanu.
\newline
\newline
Algorytm działa w sposób dynamiczny. W przypadku działania algorytmu A* na grafie - z góry znane są krawędzie oraz wierzchołki. W tej modyfikacji algorytm A* działą na dynamicznie liczonych stanach.
\newline
\newline
Do algorytmu należy także przekazać funkcję heurystyki, która jest zależna od danych danego stanu.
\newline
\newline
Algorytm prowadzi obliczenia aż nie osiągnięte zostaną, wcześniej przekazane warunki wygranej.
\newline
\newline
W opracowanym rozwiązaniu warunkiem wygranej jest opuszczenie przez wszystkie pojazdy skrzyżowań.
\newline
\newline
W rozwiązaniu przedstawione zostało także unikanie kolizji pojazdów na skrzyżowaniu oraz pojazdów poruszających się tym samym pasem.
\newline
\newline
Do systemu można przekazać każdy rodzaj skrzyżowania jako listę dróg wraz z ich danymi oraz informacjami o przecięciach z innnymi drogami.
\newline
\newline
Rozwiązanie oparte jest o przestrzeń dyskretną. Drogi mają konkretny rozmiar. System opiera się o działanie na krokach czasowych. Prędkość samochodów wyrażana jest w liczbie odcinków drogi na jeden krok czasowy.
\newline
\newline
W celu wizualizcji stworzny został także moduł graficzny, który jest wstanie zwizualizować każde podane skrzyżowanie. Przedstawia on także w formie zdjęć zaplanowany ruch samochodów. Każde zdjęcie odpowiada następnemu krokowi czasowemu. Na zdjęciu ostatnim wszystkie samochody opuszczają skrzyżowanie.
\newline
\newline
W rozwiązaniu jest także liczone prawdopodobieństwo kolizji w przypadku, gdy jeden z pojazdów nie zastosuje się do rozwiązania. Kolizja jest wówczas wykrywana. W przypadku niezastosowania się pojazdu do planu, kolejny plan powinien być wyliczony.
