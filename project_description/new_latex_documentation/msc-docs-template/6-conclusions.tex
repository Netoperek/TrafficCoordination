\mychapter{6}{6. Zakończenie} \label{chap:conclusions}

W pracy przedstawiona została metoda planowania ruchu drogowego z wykorzystaniem algorytmu A*
\newline
\newline
W rozwiązaniu zaimplementowana została modyfikacja algorytmu A*.
\newline
\newline
Analizując wyniki przedstawione w rozdziale 5 można wnioskować, że teza przedstawiona w rozdziale 3 jest prawdziwa. 
\newline
\newline
Porównanie opisywanej w tej pracy metody wraz z koordynacją ruchu przy użyciu świateł drogowych pokazuje, że metoda daje lepsze wyniki, jeżeli chodzi o przepustowość.
\newline
\newline
W przedstawionej metodzie, zapewnione jest także rozwiązywanie kolizji oraz bezpieczeńśtwo w przypadku niezastosowania się pojazdów do planu.
\newline
\newline
Rozwiązanie zostało zaprojektowane i uruchamiane na jednym wątku procesora. Używając więcej wątków, można liczyć plany wielowariantowe w celu szybkiej reakcji na pomyłki pojazdów
\newline
\newline
Jak pokazują wyniki, zastosowanie metody do koordynacji ruchu na skrzyżowaniu ze sobą ośmiu dróg zaoszczędza 5-10 kroków czasowych. Korzystając z metoda dla większej ilości sieci, można zaoszczędzić dużo czasu.
\newline
\newline
Z pewnością wadą metody jest złożoność obliczeniowa, która rośnie ekspotencjalnie względem liczby pojazdów znajdujących się na sieci szkrzyżowań.
