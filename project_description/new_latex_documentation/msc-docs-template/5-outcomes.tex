\mychapter{5}{5. Wyniki} \label{chap:outcomes}

\section{Wstęp}

W celu zbadania działania systemu przeprowadzone zostały następujące pomiary:

\begin{itemize}
\item pomiar czasu wykonania programu w zależności od liczby samochodów na skrzyżowaniu 
\item pomiar ilości kroków czasowych po których wszystkie samochody opuszczą skrzyżowania w zależnośći od liczby samochodów na skrzżowaniu
\end{itemize}

Pomiary zostały wykonane na dwóch sieciach dróg:
\begin{itemize}
\item skrzyżowanie ze sobą ośmiu dróg
\item skrzyżowanie ze sobą czterech dróg
\end{itemize}

Drogi na obydwu sieciach mają rozmiar 24.
\newline
\newline
Reprezentacja graficzna skrzyżowań jest przedstawiona na rysunku \ref{both-crossroads} z przykładowymi rozmieszczeniami pojazdów.
\begin{figure}[H]
    \includegraphics[width=1.0\textwidth]{2_crossroads.png}
  \caption{Skrzyżowanie czterech oraz ośmiu dróg}
  \label{both-crossroads}
\end{figure}

Pomiary zostały przeprowadzone dla następującej liczby samochodów \{2, 4, 6, 8, 10, 12\}.
\newline
\newline
Dla każdej liczby samochodów wygenerowane losowo zostało 30 położeń pojazdów przy założeniu, że żaden z pojazdów nie startuje za ostatnim skrzyżowaniem.
\newline
\newline
Pomiary zostały wykonane na następującym sprzęcie:
\begin{itemize}
\item Procesor: Intel(R) Core(TM) i7-4700MQ CPU @ 2.40GHz
\item Liczba rdzeni procesora: 8 - Obliczenia zostały wykonane na jednym wątki, czyli na jednym rdzeniu
\item 16 GB pamięci RAM
\end{itemize}

\section{Wyniki pomiarów dla skrzyżowania czterech dróg}

Zgodnie z wcześniej opisanymi założeniami dla skrzyżowania czterech dróg program został uruchomiony 30 razy dla samochodów ze zbioru \{2, 4, 6, 8, 10, 12\} co razem daje 180 uruchomień.
\newline
\newline
Wykresy przedstawiające wyniki na skrzyżowaniu czterech dróg zaprezentowane są na rysunkach \ref{execution_time_10_cars_4_roads}, \ref{execution_time_12_cars_4_roads} oraz \ref{four-roads-crossroads-timesteps}
\begin{figure}[H]
  \centering
  \includegraphics[width=0.8\textwidth]{execution_time_10_cars_4_roads.png}
  \caption{Wykres zależności czasu wykonania programu od liczby pojazdów (dla 2-10 pojazdów)}
  \label{execution_time_10_cars_4_roads}
\end{figure}
\begin{figure}[H]
  \centering
  \includegraphics[width=0.8\textwidth]{execution_time_12_cars_4_roads.png}
  \caption{Wykres zależności czasu wykonania programu od liczby pojazdów (dla 12 pojazdów)}
  \label{execution_time_12_cars_4_roads}
\end{figure}
\begin{figure}[H]
  \centering
  \includegraphics[width=0.8\textwidth]{4_roads_timesteps_2.png}
  \caption{Wykres zależności kroków czasowych od liczby pojazdów}
  \label{four-roads-crossroads-timesteps}
\end{figure}

\section{Wyniki pomiarów dla skrzyżowania ośmiu dróg}

Dla skrzyżowania ośmiu dróg też zostało przeprowadzone 180 uruchomień.
\newline
\newline
Wykresy przedstawiające wyniki na skrzyżowaniu ośmiu dróg zaprezentowane są na rysunkach \ref{execution_time_10_cars_8_roads}, \ref{execution_time_12_cars_8_roads} oraz \ref{eight-roads-crossroads-timesteps}
\begin{figure}[H]
  \centering
  \includegraphics[width=0.8\textwidth]{execution_time_10_cars_8_roads.png}
  \caption{wykres zależności czasu wykonania programu od liczby pojazdów (dla 2-10 pojazdów)}
  \label{execution_time_10_cars_8_roads}
\end{figure}
\begin{figure}[H]
  \centering
  \includegraphics[width=0.8\textwidth]{execution_time_12_cars_8_roads.png}
  \caption{wykres zależności czasu wykonania programu od liczby pojazdów (dla 12 pojazdów)}
  \label{execution_time_12_cars_8_roads}
\end{figure}
\begin{figure}[H]
  \centering
  \includegraphics[width=0.8\textwidth]{8_roads_timesteps_2.png}
  \caption{Wykres zależności kroków czasowych od liczby pojazdów}
  \label{eight-roads-crossroads-timesteps}
\end{figure}

\section{Wnioski z pomiarów na skrzyżowaniach czterech i ośmiu dróg}

Liczba samochodów ogranicza metodę
\newline
\newline
Czas wykonania rośnie ekspotencjalnie do liczby samochodów na skrzyżowaniach
\newline
\newline
Im więcej skrzyżowań tym czas wykonania jest większy
\newline
\newline
Dla 12 pojazdów czas wykonania przekracza nawet 25 min
\newline
\newline
Liczba kroków czasowych, po których wszystkie samochody opuszczą skrzyżowanie rośnie liniowo do liczby samochodów znajdujących się na skrzyżowaniu.
\newline
\newline
Liczba kroków czasowych na skrzyżowaniu ośmiu dróg jest mniejsza niż na skrzyżowaniu ośmiu dróg. Spowodowane jest to tym, że na skrzyżowaniu czterech dróg auta stoją jedno za drugim na pasach, bez wolnej przestrzeni przed sobą. W porównaniu do skrzyżowaniu ośmiu dróg, auta mogą swobodniej poruszać się na pasie i częściej wymijają kolizję na skrzyżowaniu.

\section{Prawdopodobieństwo kolizji}
Zostały przeprowadzone pomiary prawdopodobieństwa kolizji w przypadku pomyłek pojazdów.
\newline
\newline
Pomyłka polega na niezastosowaniu się pojazdu do planu
\newline
\newline
Przeprowadzone zostały symulacje pomyłek pojazdów - w jednym kroku czasowym, każdy samochód myli się z określonym prawdopodobieństem.
\newline
\newline
Zmierzone zostało prawdopodobieństwo kolizji w razie pomyłek pojazdów z następującymi prawdopodobieństwami \{0.001, 0.005, 0.05\}
\newline
\newline
Pod uwage wzięty został parametr bezpieczeństwa
\newline
\newline
Pomiary zostały wykonane dla skrzyżowania 8 dróg, na których znajdowało się 10 pojazdów. Program uruchomiony został 30 razy dla parametru bezpieczeństwa równego 0 oraz 30 razy dla parametru bezpieczeństwa równego 1.
\newline
\newline
Wyniki z parametrem bezpieczeństwa równym 0 zareprezentowane są w tabeli \ref{firstCollision}
\begin{table}[H]
    \centering
    \begin{tabular}{|c|c|c|c|}
      \hline 
      Prawdopodobieństwo pomyłki & 0.001 & 0.005 & 0.01 \\
      \hline
      Prawdopodobieństwo kolizji & 0.2 & 0.27 & 0.54 \\
      \hline
    \end{tabular} 
    \caption{Prawdopodobieństwo kolizji z parametrem bezpieczeństwa równym 0}
    \label{firstCollision}
\end{table}
Wyniki z parametrem bezpieczeństwa równym 1 zareprezentowane są w tabeli \ref{secondCollision}
\begin{table}[H]
    \centering
    \begin{tabular}{|c|c|c|c|}
      \hline 
      Prawdopodobieństwo pomyłki & 0.001 & 0.005 & 0.01 \\
      \hline
      Prawdopodobieństwo kolizji & 0 & 0 & 0.24 \\
      \hline
    \end{tabular} 
    \caption{Prawdopodobieństwo kolizji z parametrem bezpieczeństwa równym 0}
    \label{secondCollision}
\end{table}
Parametr bezpieczeństwa niweluje prawdopodobieństwo kolizji dla małych prawdopodobieństw pomyłki do zera.
\newline
\newline
Dla prawdopodobieństwa pomyłki równego 0.1, parametr bezpieczeństwa nie zapobiega całkowicie kolizjom.

\section{Porównanie wyników z rozwiązaniem ze światłami drogowymi}

W celu zweryfikowania szybkości podejścia z wykorzystaniem zmodyfikowanej wersji A* pomiary zostały porównane z rozwiązaniem polegającym na optymalizacji świateł drogowych przedstawionym w pracy [TUTAJ CYTAT PRACY SEBASTIANA]. 
\newline
\newline
Praca oparta jest o kroki czasowe. Drogi są dyskretne i są mierzone w odcinkach. Prędkość mierzona jest w ilości odcinków drogi na krok czasowy.
\newline
\newline
Rozwiązania różnią się modelami danych, które reprezentują skrzyżowania oraz znajdujące się na nich pojazdy. W jednym rozwiązaniu położenie i droga aut jest określona za pomocą grafu oraz wierzchołków grafu, przez które auta będą przechodzić. W opisywanym w tej pracy rozwiązaniu drogi są opisywane poprzez ich rozmiar oraz przecięcia z innymi drogami. Położenia samochodów natomiast poprzez numer drogi oraz numer odcinka drogi na której się znajdują.
\newline
\newline
Wykonując pomiary uzgodnione zostały wspólne warunki w obu rozwiązanich tak, aby można było porównać rozwiązania. Porównywane są korki czasowe, po których wszystkie auta opuszczą skrzyżowanie. Wspólne warunki są następujące:
\begin{itemize}
\item Maksymalna prędkość samochodów wynosi 4 odcinki drogi na krok czasowy
\item Wartości przyspieszeń pojazdów są następujące: \{-1, 0, 1\}
\item Losowane położenia pojazdów w celu wykonania 30 pomiarów są wykonywane w taki sposób, aby żaden pojazd nie znajdował się za skrzyżowaniem
\item Rozwiązanie kończy obliczenia, kiedy wszystkie pojazdy znajdą się za skrzyżowaniem
\end{itemize}
Pomiary zostały wykonane na skrzyżowaniu ośmiu dróg dla następujących liczb pojazdów \{2, 4, 6, 8, 10 ,12\}
\newline
\newline
Wykres z wynikami pokazany jest na rysunku \ref{comparison}
\begin{figure}[H]
  \includegraphics[width=1.0\textwidth]{8_roads_comparison_2.png}
  \caption{Porównanie rozwiązania z użyciem A* ze światłami drogowymi na skrzyżowaniu ośmiu dróg}
  \label{comparison}
\end{figure}
Rozwiązanie wypada lepiej na tle rozwiązania ze światłami sekwencyjnymi oraz z ich optymalizacjami.
\newline
\newline
Zgodnie z przedstawioną tezą, przepustowość na skrzyżowaniach jest lepsza w rozwiązaniu A* w porównaniu do świateł.
\newline
\newline
Rozwiązanie korzystające z algorytmu A* dla 12 pojazdów nie przekracza 15 kroków czasowych. Dla świateł zoptymalizowanych przekracza 20 kroków czasowych, a dla świateł sekwencyjnych przekracza 25 kroków czasowych.
\newline
\newline
Przedstawione w pracy rozwiązanie zoaszczędza 5-10 kroków na jednym skrzyżowaniu dla 12 pojazdów. Dla wielu skrzyżowań można zaoszczędzić dużo czasu.
