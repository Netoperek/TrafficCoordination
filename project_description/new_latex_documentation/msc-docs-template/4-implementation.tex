\chapter{Implementacja} \label{chap:implementation}

\section{Opis modelu danych do reprezentacji skrzyżowań}

Drogi są reprezentowane w sposób dyskretny. Każda droga zaczyna się w pozycji 1 i rośnie wraz z jej rozmiarem. 
\newline
Samochody mogą poruszać się w następujących kierunkach:
\newline
- z północy na południe
\newline
- z południa na północ
\newline
- ze wschodu na zachód
\newline
- z zachodu na wschód
\newline
\newline
Droga jest reprezentowana poprzez następujące dane:
\newline
- unikalny numer drogi
\newline
- rozmiar drogi
\newline
- informacja o przecięciach z innymi drogami
\newline
- kierunek w którym samochód się porusza na drodze (z północy na południe, czy z południa na północ lub z zachodu na wschód czy ze wschodu na zachód)
\newline
\newline
Samochody na drogach opisują następujące dane:
\newline
- unkialny numer samochodu
\newline
- unikalny numer drogi, na której się znajduje
\newline
- numer pozycji na drodze, na której samochód się znajduje
\newline
- prędkość początkowa
\newline
- numer pozycji docelowej na drodze - punkt za ostatnim skrzyżowaniem
\newline
\newline
Samochody poruszają się po drogach w krokach czasowych. Prędkość samochodu wyrażana jest w liczbie odcinków drogi w jednym kroku czasowym.
\newline
W Systemie można wybrać maksymalne przyspieszenia ujemne oraz dodatnie samochodów z następujących możliwości \{-2, -1, 0, 1, 2\}

\section{Generyczny, wielostanowy algorytm A*}

Zaprezentowny w tej pracy zmodyfikowany algorytm A* jest algorytmem wielostanowym. Oznacza to, że wierzchołkiem w grafie jest stan wszystkich samochodów na skrzyżowaniu, a krawędzią jest przejście z jednego stanu do następnego.
\newline
Algorytm ma za zadanie doprowadzić do zadanego celu otrzymując określone dane początkowego. Stanem początkowym jest startowe rozstawienie aut na drogach.
\newline
Algorytm otrzymując dane startowe, ma osiągnąć zadany cel. Danymi startowymi jest początkowe rozstawienie samochodów na skrzyżowaniach wraz z ich początkowymi prędkościami.
\newline
Modyfikacja Algorytm A* zawiera element generyczny. Do algorytmu przekazujemy klasę reprezentującą wierzchołek, która spełnia następujące wymagania:
\newline
- implementuje metodę 'neighbours', która zwraca sąsiednie wierzchołki 
\newline
- 'edge\_weight(neighbour\_vertex)', która zwraca wagę krawędzi do wierzchołka przyjmowanego w parametrze metody
\newline
\newline
Do algorytmu przekazuje się także warunki wygranej. W przypadku mojego rozwiązania wygraną jest przekroczenie przez wszystkie samochody ostatniego skrzyżowania na drogach, na których się znajdują.
\newline
Do algorytmu przekazywana jest także funkcja heurystyki, zależna od danych danego wierzchołka.

\section{Reprezentacja wierzchołka w zmodyfikowanym algorytmie A*}

\section{Unikanie kolizji}

\section{Reprezentacja i wczytywanie danych}

\section{Reprezentacja graficzna wyników}
