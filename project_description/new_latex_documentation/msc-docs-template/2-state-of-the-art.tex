\chapter{Istniejące rozwiązania w zakresie planowania ruchu i koordynacji ruchu na skrzyżowaniach} \label{chap:state-of-the-art}

\section{Planowanie ruchu na skrzyżowaniach}

\textbf{Cel - lista wymagań i cech mojego rozwiązania - tego nie będzie w treści}
\newline
- Zastosowanie algorytmu A* do zaplanowania ruchu na skrzyżowaniach
\newline
- Brak świateł powinno przyspieszyć ruch na szkyżowaniach - auta nie czekają na świetle pomarańczowym
\newline
- Rozwiązanie uwzględnia unikanie kolizji
\newline
- A* powinien szybciej zaplanować drogę niż np. Dijkstra
\newline
- W rozwiązaniu zapewnione jest bezpieczeństwo - nie dochodzi do kolizji
\newline
- Rozwiązanie koordynuje ruch globalnie
\newline
\newline
\newline
\newline
\indent
  Planowanie ruchu zawsze wiąże się z rozmiarem problemu. Dla koordynacji ruchu na skrzyżowaniach rozmiarem problemu jest rozmiar dróg, ich liczba oraz liczba przecięć między nimi, a także parametry pojazdów. Z planowaniem ruchu wiąże się trudność jaką jest szybkość zmian w środowisku. W przypadku nie zastosowania się pojazdu do planu, algorytm musi natychmiast wyliczyć poprawiony plan. Z tego względu algorytm nie powinien zajmować dużo czasu. Algorytmy heurystyczne są najczejściej stosowane w często zmieniających się środowiskach.
\newline
\indent
  Często stosowany algorytmem do znajdowania najkrótszej ścieżki w grafie jest algorytm Dijkstry. O rzędzie złożoności algorytmu decyduje implementacja kolejki. W przypadku implementacji kolejki poprzez kopiec złożoność algortmu to:
\newline
\newline
\begin{math}
O(E + V\log V)
\end{math}
\newline
\newline
gdzie \begin{math}E\end{math} to liczba krawędzi grafu, a \begin{math}V\end{math} liczba wierzchołków grafu.
\indent
\newline
\newline
\indent
  Algorytmem heurystycznym stosowanym do szukania najkrótszej ścieżki jest Algorytm A*. Złożoność czasowa algorytmu jest zależna od zastosowanej funkcji heurystyki. Jeśli funkcja heurystyki spełnia następujący warunek:
\newline
\newline
\begin{math}
|h(x)-h^{*}(x)|=O(\log h^{*}(x))
\end{math}
\newline
\newline
gdzie \begin{math}h*\end{math} jest optymalną heurystyką - liczba przeszukanych węzłów rośnie wielomianowo w stosunku do długości rozwiązania.
\newline
\newline
\indent
W artykule \cite{wojnicki2015robust} autorzy przedstawili algorytm A* w celu planowania wielowymiarowego. W celu przybliżenia złożoności jako przykładowe środowisko podana została dwuwymiarowa przestrzeń kwadratowa o boku 7. Pojazdy mogą zajmować wówczas 49 pozycji oznaczonych współrzednymi \begin{math}(x, y)\end{math}. Dla jednego pojazdu, możliwość zajmowanych pozycji wynosi \begin{math}(7 x 7) = 49 \end{math}. Dla dwóch pojazdów jest to już \begin{math}(7 x 7) x (7 x 7 - 1) = 2352 \end{math}. Aktualny stan jest wówczas opisywany jako \begin{math}(x1, y1, x2, y2)\end{math}. Złożoność rośnie wraz z przetrzymywaniem kolejnych informacji na temat pojazdów, czyli  na przykład prędkośći czy przyspieszenia.
\indent
\newline
Autorzy artykułu \cite{leena2014survey} zwrócili uwagę na podział planowania na globalne i lokalne. W podejśćiu globalnym dla algorytmu musi być znane całe środowisko co wiąże się z dużym obciążeniem pamięciowym - algorytm przechowuje informacje na temat stanów wszystkich pojazdów na skrzyżowaniach. W podejśćiu lokalnym środowisko dla pojedyńczego pojazdu nie jest znane, algorytm poznaje je w czasie rzeczywistym. W przedstawionym rozwiązaniu zastosowane jest podejście globalne.

\section{Planowanie ruchu przy użyciu świateł drogowych}

\indent
Najczęściej stosowanym systemem do koordynacji ruchu na skrzyżowaniach są światła drogowe. Powodują one opóźniena wiążące się z oczekiwaniem przy świetle pomarańczowym, które jednocześnie zapewnia bezpieczeństwo. Powstało wiele optymalizacji świateł drogowych w celu pryspieszenia ruchu na skrzyżowaniach.
\newline
\indent
W artykule \cite{brockfeld2001optimizing} autorzy opisują następujące optymalizacje sygnalizacji świetlnej: \textit{Synchronized Traffic Lights, Green Wave, Random Offset}. Synchroniczne światła drogowe oparte są o obliczane cykle oraz o klastry samochodow poruszających się po skrzyżowaniach. Klaster samochodów rusza, gdy pojawi się światło zielone. Pierwszy samochód rusza z największym przyspieszeniem, aż dojedzie do następnego skrzyżowania. Wówczas natychmiast włącza się światło czerwone i reszta samochodów w klastrze dojeżdża do skrzyżowania. Cykle są obliczane w sposób, aby opóźnienia na skrzyżowaniach były najmniejsze. Zgodnie ze zdaniem autorów, jedną z wad jest fakt, że rozwiązanie działa najlepiej dla nierealistycznych cykli. W celu usprawnienia autorzy zaproponowali strategię zielonej fali. Głównym celem tego podejście jest to, aby duże klastry samochodów były w ruchu. W artykule, strategia zielonej fali daje lepsze efekty niż startegia synchronicznych świateł. Ostatnią przedstawioną strategią jest losowa zmiana świateł przed skrzyżowaniami. W przeciwieństwie do poprzednich strategii - światła na skrzyżowanach nie przełączją się jednocześnie. Użycie tej strategii powoduje powstawanie dużych klastrów samochodów na niektórych drogach i szybki przepływ na innych. Globalnie strategia zielonej falii przynosiła najlepsze efekty. Przedstawione rozwiązania muszą jednak stosować światło w celu utrzymania bezpieczeństwa co powoduje opóźnienia na każdym skrzyżowaniu.
\newline
\indent
Autorzy artykułu \cite{athmaraman2005adaptive} opisują koordynację ruchu przy użyciu algorytmu 'REAL TIME QUEUE LENGTH ESTIMATION: THE APTTC ALGORITHM'. Koordynacja ta działa przy użyciu tablicy sensorów fotoelektrycznych rozmieszczonych w topologii. Algorytm w zależności od danych i estymacji kolejek odpowowiednio steruje sygnalizacją świetlną w celu uniknięcia opóźnień. Algorytm został uruchomiony w mieście Chennai w Indiach, gdzie wyestymowane zostały długości kolejek aut na drogach poprzez zamontowane sensory na każdy dzień tygodnia. Rozwiązanie jest jednak zaprojektowane do działania na jednym skrzyżowaniu oraz nie eliminuje opóźnień spowodowachnych światłem pomarańczowym. Co więcej, w rozwiązaniu konieczne jest montowanie sensorów na drogach.
\newline
\indent
W artykule \cite{de1998optimal} opisane zostało kierowanie ruchem drogowym na jednym skrzyżowaniu poprzez optymalizację kolejek. W rozwiązaniu brane pod uwage są takie dane jak: średni czas oczekiwania, średnia długość kolejki, najgorszy czas oczekiwania. Istnieje także możliwość przekazania parametrów odpowiadających za maksymalny czas trwania światła czerwonego oraz zielonego. Wynikiem rozwiązania jest schemat zmiany świateł, który liczony jest za pomocą funkcji obiektowych na podstawie długości kolejek. W rozwiązaniu pominięty został element bezpieczeństwa jakim jest światło pomarańczowe. Rozwiązanie zostało pokazane także na jednym skrzyżowaniu, a nie na całej sieci skrzyżowań.
\newline
\indent
  Autorzy artykulu \cite{lammer2008self} przedstawili system kontroli świateł i przepływu pojazdów na sieciach drogowych. Rozwiązanie stosowane jest dla całych sieci skrzyżowań. Autorzy opierają się o krótkie prognozu ruchu i na ich podstawie liczą długość trwania światła zielonego. Klasyczne rozwiązania działają na wcześniej obliczonych danych oraz działają we wcześniej określonych miejscach największego ruchu. Takie rozwiązania powodują, że zielona fala trwa dużej niż powinna. W rozwiązaniu zastosowana jest także strategia stabilizacji, która zapewnia fakt, iż długość kolejek nie będzie zbyt długa. Zapewnione jest to przez odpowiednią długość trwania światła zielonego. W rozwiązaniu, na każdym ze skrzyżowań, samochody będą tracic czas na świetle pomarańczowym, które w rozwiązaniu jest konieczne, aby było ono bezpieczne. Ruch jest także optymalizowany dla sytuacji "średnich", które tak na prawdę nigdy nie występują. Wynikiem tego zjawiska jest fakt, że optymalizacja nie jest nigdy stosowana dla prawdziwej, aktualnej sytuacji.
\newline
\indent
\textbf{Zarządzanie światłami za pomocą komunikacji pojazdów między sobą \cite{ferreira2010self}}
\newline
Wszystkie auta muszą być wyposażone w:
\newline
	- Urządzenia DSRC
\newline
  - Tą sama wersję mapy
\newline
	- GPS z dokładnością do pasa na którym są
\newline
Sieć bezprzewodowa w każdym z samochdów musi być niezawodna
\newline
\newline
\textbf{Krytyka}
\newline
Rozwiązanie w porównaniu do mojego jest kosztowne, jeżeli chodzi o wyposażenie. Jest ono możliwe także tylko wtedy kiedy wszystkie auta są odpowiednio wyposażone i sprawne.
    
\section{Planowanie ruchu bez sygnalizacji świetlnej}

W celu uniknięcia użycia świateł drogowych istnieją także rozwiązania z wykorzystaniem algorytmów znajdujących najkrótszą ścieżkę. W artykule \cite{shaikh2013agv} autorzy wykorzystali algorytm Dijkstry do zarządzania ruchem przy użyciu automatycznie prowadzonych pojazdów. Autorzy przedstawili zarządzanie ruchem pojazdów w magazynie chcąc ogarniczyć czas transportu. Algorytm Dijkstry wskazuje najkrótszą ścieżkę od jednego pojazdu do następnego. W przypadku, gdy na ścieżce znajdzie się przeszkoda - algorytm wylicza ścieżkę od nowa. Zastosowanie algorytmu Dijkstry do koordynacji ruchu na skrzyżowaniach jest zbyt złożone. Ponadto unikanie kolizji jest wykonane poprzez ponowne wyliczanie najkrótszej ścieżki, zamiast unikania ich na bieżąco.
\newline
\indent
Autorzy artykułu \cite{huang2013improved} przedstawiają ulepszoną wersję algorytmu Dijkstry w celu znalezienia najkrótszej ścieżki. Do algorytmu Dijkstry wprowadzona została funkcja dzięki której ignorowane są nieistotne wierzchołki w grafie. Dzięki tej modyfikacji algorytm jest jest ~43\%-76\% szybszy. Rozwiązanie jest nadal zbyt złożone by zastosować go do koordynacji ruchu na skrzyżowaniach. W rozwiązanie nie ma także uwzględnionego unikania kolizji.
\newline
\indent
W artykule \cite{ando2003autonomous} przedstawione zostało planowanie ruchu przy użyciu automatycznie prowadzonych pojazdów wraz z unikaniem kolizji. W rozwiązaniu każdy robot indywidualnie liczy najkrótszą dla siebie ścieżkę. Następnie dane tras są wymieniane pomiędzy robotami w celu ustalenia finalnego rozwiązania. Szukanie ścieżki jest wykonywane przy użyciu algorytmu Dijkstry. W celu unikania kolizji dla każdego pojazdu liczona jest funkcja kary. Waga kary jest zwiększa tak długo, aż zostanie znaleziona bezkolizyjna ścieżka. Złożoność rozwiązania jest zbyt duża by zastosować je dla skrzyżowań. Wymienianie danych pomiędzy pojazdami oraz podany w rozwiązaniu sposób rozwiązywania kolizji nie znajduje zastosowania w koordynowaniu ruch na skrzyżowaniach.
\newline
\indent
W patencie \cite{gazis1997optimal} autorzy przedstawili centralnie zaplanowany, generalny system przydzielający trasy w celu utrzymania optymalnego ruchu. System zawiera dużą ilość komputerów zamontowanych w pojazdach w celu komunikacji między nimi poprzez sieć bezprzewodową. Do znajdownia najkrótszej ścieżki autorzy wykorzystali algorytm Dijkstry. Ścieżki liczone są przy użyciu daynch dostarczanych przez komputery zamontowane w pojazdach. Celem patentu jest znalezienie drogi do celu w najkrótszym czasie poprzez omijanie zakorkowanych dróg. Nie jest to dokładne zaplanowanie poruszania się pojazdu wraz z unikaniem kolizji.
\newline
\indent
Autorzy patentu \cite{broxmeyer1994vehicle} opisali system do controli ruchu i unikania kolizji na autostradach. Pod uwage wzięte zostały zmiany pasów, unikanie kolizji oraz kontrolowanie trasy pojazdu. Unikanie kolizji zostało zapewnione poprzez komunikacje przez transmitery radiowe oraz radioodbiorniki zamontowane w każdym z pojazdów. Rozwiązanie znajduje zastosowanie na autostradach, unikanie kolizji ma miejsce na równoległych pasach. Tego podejścia nie da się zastosować jeżeli chodzi o koordynacje ruchu na skrzyżowaniach.
\newline
\indent
W artykule \cite{kanoh2007dynamic} przedstawiony został system dynamicznego planowania ruchu do nawigacji samochodów przy użyciu 'virus genetic algorithms'. Autorzy zaproponowali algorytm genetyczny, który jest bardziej wydajny od algorytmu A* czy algorytmu Dijkstry. W rozwiązaniu zastosowana jest strategia infekcji wirusowej, dzięki czemu omijane są korki na drogach. Rozwiązanie jest zaprojektowane do wskazania najlepszej trasy, nie jest to dokładne zaplanowanie ruchu samochodów wraz z wymijaniem kolizji.

\section{Użycie algorytmu A* do planowania ruchu}

\indent
Algorytm A* jest często stosowany do szukania najkrótszej ścieżki oraz planowania ruchu. W artykule \cite{dechter1985generalized} omówiona została optymalność tego algorytmu oraz funkcja heurystyki. A* jest algorytmem heurystycznym, który znajduje najkrótszą ścieżkę w grafie ważonym z dowolnego podanego wierzchołka do wierzchołka celu. Wierzchołkiem celu jest wierzchołek spełniający zadane warunki wygranej. W artykule przedstawione zostały poprzednie próby udowodnienia optymalności algorytmu. Autorzy przeprowadzili także własny dowód na optymalność algorytmu A*. Dowód polega na fakcie, iż kiedy funkcja heurystyki nigdy nie przeszacowuje wtedy algorytm A* jest optymalny. Dla grafu, algorytm A* od zadanego wierzchołka startowego tworzy ścieżkę, wybierając niezbadany sąsiedni wierzchołek. Wybiera go w sposób aby minimalizować funkcję:
\newline
\newline
\begin{math} f(x) = g(x) + h(x)\end{math}
\newline
\newline
gdzie:
\newline
\newline
\begin{math} g(x) \end{math} - suma wag krawędzi od wierzchołka startowego do wierzchołka x
\begin{math} h(x) \end{math} - przewidywana przez funkcję heurystyki suma wag z wierzchołka x do wierzchołka docelowego
\newline
\newline
\textbf{Omówienie i porównanie agorytmów stosowanych do planowania drogi na podstawie artykułu ~\cite{delling2009engineering}}
  \newline
  - Dijkstra's Algorithm
  \newline
  - Priority Queues
  \newline
  - Bidirectional Search
  \newline
  - A*
\newline
\newline
\textbf{Planowanie ruchu za pomocą zmodyfikowanego algorytmu A* na podstawie artykułu ~\cite{munteanmobile}}
  \newline
  - Opis sposobu wprowadzenia zmodyfikowanego algorytmu A* do koordynowania ruchu na skrzyżowaniach
  \newline
  - Wady, zalety i ogarniczenia
\newline
\newline
\textbf{Planowanie ruchu za pomocą wielostanowego algorytmu A* oraz Wavefront na podstawie artykułu ~\cite{wojnicki2015robust}}
\newline
- Omówienie zastosowania tych dwóch algorytmów do planowania ruchu
\newline
\textbf{Krytyka}
\newline
Złożoność użycia wielostanowego A* jest już duża do planowania ruchu na skrzyżowaniach - dodając do tego wavefront złożoność jest jeszcze większa.
\newline
\newline
\textbf{Omówienie przykładu użycia Multi-Entity A* na podstawie artykułu ~\cite{wojnicki2015robust}}
  \newline
  - Przytoczenie przykładu
  \newline
  - Omówienie wad, zalet i ogarniczeń
  \newline
  \newline
Omówienie złożoności Multi-Entity A* na powyższym przykładzie.
  \newline
  - Złożoność jest zależna od liczby stanów w A*
  \newline
  - Obliczenie złożoności i pokazania jak ona rośnie
\newline
\newline
\textbf{Autorzy artykułu ~\cite{elhalawany2013modified} zaproponowali użycie zmodyfikowanego algorytmu A* w celu bezpieczniejszej nawigacji dla robotów}
  \newline
  - Modyfikacja algorytmu polega na wzięciu pod uwagę rozmiaru robota w celu generowania prostrzej ścieżki i w celu uniknięcia ostrych skrętów
  \newline
  - Do algorytmu A* przekazywany jest stan początkowy i cel jaki robot chce osiągnąć
  \newline
  - Autorzy zmodyfikowali algorytm A* tak aby unikał niebezpiecznych sytuacji, które mogą powodować kolizje
\newline
\newline
\textbf{Krytyka}
\newline
 Algorytm jest liczony dla jednego robota - u mnie brane są pod uwage wszystkie pojazdy na drogach

\section{Modyfikacje A*}
\textbf{Cel podrozdziału}
\newline
Przytoczenie modyfikacji algorytmu A* w literaturze. Wskazanie wad takich modyfikacji w porównaniu do mojej
\newline
\newline
\textbf{Autorzy artykułu ~\cite{wojnicki2015robust} przedstawili wielostanowy A* }
\newline
- Ścieżka powinna być rozumiana jako sekwencja stanów
\newline
- Autorzy zastosowali modyfikację wraz z algorytmem Wavefront
\newline
\newline
\textbf{Krytyka}
\newline
- Zastosowanie tych dwóch algorytmów do rozwiązania skrzyżowań wiąże się ze zbyt dużą złożonością
\newline
\newline
\textbf{Opis modyfikacji A* w artykule ~\cite{munteanmobile}}
\newline
- Prezentacja szybkiego algorytmu A* w celu nawigacji robota
\newline
- Autorzy użyli innych struktur w implementacji algorytmu A* co daje lepsze wyniki
\textbf{Krytyka}
\newline
- Rozwiązanie nie może być zastosowane do koordynacji ruchu na skrzyżowaniach - jest ono zaprojektowane do znalezienia najkrótszej ścieżki dla pojedyńczego pojazdu
\newline
\newline
\textbf{W artykule ~\cite{oleiwi2014modified} autorzy zaproponowali genetyczny algorytm oparty na zmodyfikowanym algorytmie A* w celu optymalizacji znajdowania ścieżek dla wielu obiektów}
\newline
- Znajdowanie śecieżek dla robotów z wymijaniem przeszkód
\newline
- Autorzy zaproponwali modyfikację A* w celu znalezienia suboptymalnej ścieżki w celu ustalenia początkowego rozwiązania dla algorytmu genetycznego.
\newline
- Modyfikacja została wykonana, aby A* nie wyszukiwał najkrótszej ścieżki na rzecz wydajności
\newline
\newline
\textbf{Krytyka}
\newline
Nie jest to jednak Algorytm A* który opiera się na stanie wszystkich pojazdów w środowisku - polega on na znalezienie ścieżki dla jednego robota.
