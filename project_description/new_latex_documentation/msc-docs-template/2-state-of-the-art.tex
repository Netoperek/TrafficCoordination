\chapter{State of the art} \label{chap:state-of-the-art}

\section{Planowanie ruchu na skrzyżowaniach}

\textbf{Cel podrozdziału}
\newline
Wskazanie złożoności problemu znajdowania drogi oraz wskazania algorytmów używanych w literaturze do tego typu probemów
\newline
\newline
\textbf{Zależność wydajności od rozmiaru problemu}
\newline
- Metody planujące są zawsze zależne od rozmiaru problemu
\newline
- Duża złożoność powoduje stosowanie algorytmów heurystycznych
\newline
- Pokazanie złożoności na przykładowych algorytmach
\newline
\newline
\textbf{Szybkość zmian w środowisku uniemożliwia użycia algorytmu zajmującego dużo czasu}
\newline
- Duża złożoność problemu powoduje, że jeżeli zajdzie zmiana w środowisku, algorytm musi natychmiast wyliczyć poprawny plan
\newline
- Algorytmy heurystyczne są wstanie działać w środowisku, gdzie zmiany zachodzą szybko
\newline
\newline
\textbf{Rozważana klasa problemów}
\newline
- Omówienie klasy problemów planowania ruchu na skrzyżowań
\newline
- Omówienie modelów środowisk i wiążących się z nimi złożoności
\newline
\newline
\textbf{Zgodnie z artykułem ~\cite{leena2014survey} szukanie ścieżki można podzielić na planowanie globalne oraz lokalne}
\newline
- W globalnym dla algorytmu znane musi być całe środowisko
\newline
- W lokalnym dla jednego pojazdu całe środowisko jest nie znane - jest poznawane w czasie rzeczywistym
\newline
- W mojej pracy zastosowany będzie algorytm globalny
\newline
\newline

\section{Planowanie ruchu przy użyciu świateł drogowych}

\textbf{Cel podrozdziału}
\newline
Przedstawienie rozwiązań koordynujących ruch drogowy przy użyciu świateł drogowych, omówienie ich optymalizacji oraz wskazanie co w nich powoduje opóźnienie, którego w moim rozwiązaniu nie ma.
\newline
\newline
\textbf{W artykule ~\cite{brockfeld2001optimizing} opisane zostały optymalizacje sygnalizacji świetlnej}
  \newline
  - Synchronized Traffic Lights
  \newline
  - Green Wave
  \newline
  - Random Offset
  \newline
  \newline
  - Która optymalizacja co wnosi
  \newline
  - Jak niewelowane są opóźnienia
  \newline
  - Która z nich daje najlepsze wyniki
  \newline
  \newline
\textbf{Krytyka}
\newline
Opóźnienia i tak są spowodowane przez użycie świateł (światło pomarańczowe) - te opóźnienia nie występują w moim rozwiązaniu
\newpage
\textbf{Autorzy artykułu ~\cite{athmaraman2005adaptive} opisują koordynację ruchu przy użyciu algorytmu - 'REAL TIME QUEUE LENGTH ESTIMATION: THE APTTC ALGORITHM'}
  \newline
  - Sterowanie adaptacyjne
  \newline
  - Statystyczna optymalizacja
  \newline
  - Estymacja na podstawia długości dynamicznej kolejki
\newline
\newline
\textbf{Krytyka}
\newline
Rozwiązanie nie bierze pod uwagę bezpieczeństwa, które w zwykłej sygnalizacji świetlnej jest zapewnione przez swiatło pomarańczowe.
\newline
\newline
\textbf{Opis kierowania ruchem drogowym na jednym skrzyżowaniu. Autorzy artykułu \cite{de1998optimal} dla ułatwienia pomijają światło pomarańczowe}
  \newline
  - Optymalizacja świateł za pomocą dwóch podanych funkcji polegających na optymalicazji kolejek na czterech pasach
\newline
\newline
\textbf{Krytyka}
\newline
Rozwiązanie nie bierze pod uwagę bezpieczeństwa, które w zwykłej sygnalizacji świetlnej jest zapewnione przez swiatło pomarańczowe.
\newline
\newline
\textbf{Kontrola ruchu drogowego za pomocą świateł. Autorzy atykułu \cite{lammer2008self} zaproponowali zdecentralizowany algorytm bazujący na krótkich prognozach ruchu}
  \newline
  - Optymalizacja świateł za pomocą dwóch podanych funkcji polegających na optymalizacji kolejek na czterech pasach
  \newline
	- Liczona długość światła zielonego w celu zwolnienia kolejki na pasie
  \newline
  - Limitacja - Zielone światła są zapalone dłużej niż zwykle powinny
\newline
\newline
\textbf{Krytyka}
\newline
Limitacja - Ruch jest optymalizowany dla sytuacji "średnich", które tak na prawdę nigdy nie występują, przez co optymalizacja nie jest stosowana dla aktualnej sytuacji
\newpage
\textbf{Zarządzanie światłami za pomocą komunikacji pojazdów między sobą \cite{ferreira2010self}}
\newline
Wszystkie auta muszą być wyposażone w:
\newline
	- Urządzenia DSRC
\newline
  - Tą sama wersję mapy
\newline
	- GPS z dokładnością do pasa na którym są
\newline
Sieć bezprzewodowa w każdym z samochdów musi być niezawodna
\newline
\newline
\textbf{Krytyka}
\newline
Rozwiązanie w porównaniu do mojego jest kosztowne, jeżeli chodzi o wyposażenie. Jest ono możliwe także tylko wtedy kiedy wszystkie auta są odpowiednio wyposażone i sprawne.
    
\section{Planowanie ruchu bez sygnalizacji świetlnej}

\textbf{Cel podrozdziału}
\newline
Podanie metod w literaturze do planowania drogi, koordynowania ruchu. Wskazanie wad tych rozwiązań w porównaniu do mojego.
\newline
\newline
\textbf{Autorzy artykułu ~\cite{shaikh2013agv} opisali znajdowanie ścieżki za pomocą algorytmu Dijkstry}
\newline
- Opisanie sposobu zastosowanie dijkstry w planowaniu ruchu
\newline
- Opisanie sposobu unikania kolizji
\newline
- Ograniczenia algorytmu
\newline
\newline
\textbf{Krytyka}
\newline
- Algorytm A* jest lepszy wydajnościowy względem Dijkstry
\newline
\newline
\textbf{Autorzy artykułu ~\cite{huang2013improved} przedstawiają ulepszoną wersję algorytmu Dijkstry w celu znalezienia najkrótszej ścieżki}
\newline
- Algorytm Dijkstry ma słabą wydajność, dlatego postanowiono go zmodyfikować. Według badań przeprowadzonych przez autorów algorytm jest ~42\%-76\% szybszy
\newline
- Autorzy wspominają o algorytmie A*, który jest szybszy od algorytmu Dijkstry ale może skończyć w nieskończonej pętli
\newline
\newline
\textbf{Krytyka}
\newline
- W prowadzonych przeze mnie badaniach algorytm A* ani raz nie został wprowadzony w nieskończoną pętlę oraz jest szybszy od algorytmu Dijkstry w szukaniu ścieżki.
\newline
\newline
\textbf{Planowanie ruchu oraz unikanie kolizji przy użyciu AGV na podstawie artykułu ~\cite{ando2003autonomous}}
\newline
- Opis zastosowanego rozwiązania
\newline
- Opis sposobu unikania kolizji
\newline
- Omówienie przykładu z użyciem AGV na podstawie artykułu
\newline
\newline
\textbf{Krytyka}
\newline
Zastosowanie AGV jest przeznaczone dla robotów. Jest tu także komunikacja między pojazdami co wprowadza dodatkową warstwę trudności jeżeli chodzi o wprowadzenie rozwiązania dla koordynacji ruchu na skrzyżowaniach.
\newline
\newline
\textbf{Omówienie i porównanie agorytmów stosowanych do planowania drogi na podstawie artykułu ~\cite{delling2009engineering}}
  \newline
  - Dijkstra's Algorithm
  \newline
  - Priority Queues
  \newline
  - Bidirectional Search
  \newline
  - A*
  \newline
\newline
\newline
\textbf{Autorzy artykułu ~\cite{gazis1997optimal} przedstawiają system planujący ruch pojazdów połączonych za pomocą bezprzewodowej komunikacji. System dostarcza pojazdą optymalną trasę biorąc pod uwagę aktualny ruch na drogach}
\newline
- Użycie algorytmu Dijkstry w celu znalezienia najkrótszej ścieżki
\newline
- Celem patentu znalezienie drogi do celu w najkrótszym czasie poprzez omijanie zakorkowanych dróg
\newline
\newline
\textbf{Krytyka}
\newline
- Jest to jedynie znajdowanie najszbszej drogi do celu poprzez komunikacje bezprzewodową. Nie jest to dokładne zaplanowanie ruchu pojazdu wraz z unikaniem kolizji
\newline
\newline
\textbf{W Artykule ~\cite{broxmeyer1994vehicle} autorzy przedstawili system do controli ruchu i unikania kolizji na autostradach}
\newline
- Pod uwage wzięte zostały: zmiany pasów, unikanie kolizji, kontrolowanie trasy pojazdu
\newline
- Unikanie kolizji jest zapewniane poprzez komunikacje przez transmitery radiowe oraz radioodbiorniki zamontowane w każdym z pojazdów
\newline
\newline
\textbf{Krytyka}
\newline
System planuje trasy oraz zmiany pasów na autostradach przy czym zapewnione jest bezpieczeństwo kolizji. Rozwiązania nie można jednak zaaplikować do skrzyżowań dróg
\newline
\newline
\textbf{Autorzy artykułu ~\cite{kanoh2007dynamic} przedstawiają system dynamiczny system planowania ruchu do nawigacji samochodów przy użyciu 'virus genetic algorithms'}
\newline
Zaproponowany został 'Genetic Algorithm', który powinien dawać lepsze wyniki w porównaniu do algorytmu Dijkstry oraz algorytmu A*
\newline
\newline
\textbf{Krytyka}
\newline
Brak unikania kolizji

\section{Użycie algorytmu A* do planowania ruchu}

\textbf{Cel podrozdziału}
\newline
Wskazania użyć A* w literaturze do planowania ruchu. Czy można użyć A* do planowania ruchu na skrzyżowaniach i wady użycia w porównania do mojego rozwiązania.
\newline
\newline
\textbf{Omówienie algorytmu A* na podstawie artykułu ~\cite{dechter1985generalized}}
  \newline
  - Omówienie optymalności
  \newline
  - Omówienie funkcji heurystyki
\newline
\newline
\textbf{Planowanie ruchu za pomocą zmodyfikowanego algorytmu A* na podstawie artykułu ~\cite{munteanmobile}}
  \newline
  - Opis sposobu wprowadzenia zmodyfikowanego algorytmu A* do koordynowania ruchu na skrzyżowaniach
  \newline
  - Wady, zalety i ogarniczenia
\newline
\newline
\textbf{Planowanie ruchu za pomocą wielostanowego algorytmu A* oraz Wavefront na podstawie artykułu ~\cite{wojnicki2015robust}}
\newline
- Omówienie zastosowania tych dwóch algorytmów do planowania ruchu
\newline
\textbf{Krytyka}
\newline
Złożoność użycia wielostanowego A* jest już duża do planowania ruchu na skrzyżowaniach - dodając do tego wavefront złożoność jest jeszcze większa.
\newline
\newline
\textbf{Omówienie przykładu użycia Multi-Entity A* na podstawie artykułu ~\cite{wojnicki2015robust}}
  \newline
  - Przytoczenie przykładu
  \newline
  - Omówienie wad, zalet i ogarniczeń
  \newline
  \newline
Omówienie złożoności Multi-Entity A* na powyższym przykładzie.
  \newline
  - Złożoność jest zależna od liczby stanów w A*
  \newline
  - Obliczenie złożoności i pokazania jak ona rośnie
\newline
\newline
\textbf{Autorzy artykułu ~\cite{elhalawany2013modified} zaproponowali użycie zmodyfikowanego algorytmu A* w celu bezpieczniejszej nawigacji dla robotów}
  \newline
  - Modyfikacja algorytmu polega na wzięciu pod uwagę rozmiaru robota w celu generowania prostrzej ścieżki i w celu uniknięcia ostrych skrętów
  \newline
  - Do algorytmu A* przekazywany jest stan początkowy i cel jaki robot chce osiągnąć
  \newline
  - Autorzy zmodyfikowali algorytm A* tak aby unikał niebezpiecznych sytuacji, które mogą powodować kolizje
\newline
\newline
\textbf{Krytyka}
\newline
 Algorytm jest liczony dla jednego robota - u mnie brane są pod uwage wszystkie pojazdy na drogach

\section{Modyfikacje A*}
\textbf{Cel podrozdziału}
\newline
Przytoczenie modyfikacji algorytmu A* w literaturze. Wskazanie wad takich modyfikacji w porównaniu do mojej
\newline
\newline
\textbf{Autorzy artykułu ~\cite{wojnicki2015robust} przedstawili wielostanowy A* }
\newline
- Ścieżka powinna być rozumiana jako sekwencja stanów
\newline
- Autorzy zastosowali modyfikację wraz z algorytmem Wavefront
\newline
\newline
\textbf{Krytyka}
\newline
- Zastosowanie tych dwóch algorytmów do rozwiązania skrzyżowań wiąże się ze zbyt dużą złożonością
\newline
\newline
\textbf{Opis modyfikacji A* w artykule ~\cite{munteanmobile}}
\newline
- Prezentacja szybkiego algorytmu A* w celu nawigacji robota
\newline
- Autorzy użyli innych struktur w implementacji algorytmu A* co daje lepsze wyniki
\textbf{Krytyka}
\newline
- Rozwiązanie nie może być zastosowane do koordynacji ruchu na skrzyżowaniach - jest ono zaprojektowane do znalezienia najkrótszej ścieżki dla pojedyńczego pojazdu
\newline
\newline
\textbf{W artykule ~\cite{oleiwi2014modified} autorzy zaproponowali genetyczny algorytm oparty na zmodyfikowanym algorytmie A* w celu optymalizacji znajdowania ścieżek dla wielu obiektów}
\newline
- Znajdowanie śecieżek dla robotów z wymijaniem przeszkód
\newline
- Autorzy zaproponwali modyfikację A* w celu znalezienia suboptymalnej ścieżki w celu ustalenia początkowego rozwiązania dla algorytmu genetycznego.
\newline
- Modyfikacja została wykonana, aby A* nie wyszukiwał najkrótszej ścieżki na rzecz wydajności
\newline
\newline
\textbf{Krytyka}
\newline
Nie jest to jednak Algorytm A* który opiera się na stanie wszystkich pojazdów w środowisku - polega on na znalezienie ścieżki dla jednego robota.
