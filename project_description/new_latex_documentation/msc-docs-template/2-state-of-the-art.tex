\chapter{State of the art} \label{chap:state-of-the-art}

\section{Planowanie ruchu na skrzyżowaniach}

CEL:
\newline
- wskazanie tematyki literatury jaka będzie omawiana
\newline
- jakie algorytmy w literaturze są stosowane do planowania ruchu drogowego
\newline
- algorytmy heurystyczne są preferowane ze względu na złożoność
\newline
\newline
Zależność wydajności od rozmiaru problemu
\newline
- Metody planujące są zawsze zależne od rozmiaru problemu
\newline
- Duża złożoność powoduje stosowanie algorytmów heurystycznych
\newline
- Pokazanie złożoności na przykładowych algorytmach
\newline
\newline
Szybkość zmian w środowisku uniemożliwia użycia algorytmu zajmującego dużo czasu
\newline
- Duża złożoność problemu powoduje, że jeżeli zajdzie zmiana w środowisku, algorytm musi natychmiast wyliczyć poprawny plan
\newline
- Algorytmy heurystyczne są wstanie działać w środowisku, gdzie zmiany zachodzą szybko
\newline
\newline
Rozważana klasa problemów
\newline
- Omówienie klasy problemów planowania ruchu na skrzyżowań
\newline
- Omówienie modelów środowisk i wiążących się z nimi złożoności
\newline
\newline

\section{Planowanie ruchu przy użyciu świateł drogowych}

CEL:
\newline
- omówienie literatury rozwiązującej ruch drogowy przy użyciu świateł drogowych
\newline
- światła drogowe wprowadzają opóźnienie spowodowane światłem pomarańczowym
\newline
- omówienie optymalizacji świateł drogowych, które mimo wszystko powodują opóźnienia
\newline
\newline
Opis optymalizacji wprowadzonych do zarządzania sygnalizacją świetlną na podstawie artykułu ~\cite{brockfeld2001optimizing}:
  \newline
  - Synchronized Traffic Lights
  \newline
  - Green Wave
  \newline
  - Random Offset
\newline
\newline
Porównanie powyższych optymalizacji na podstawie artykułu ~\cite{brockfeld2001optimizing}
\newline
- która optymalizacja co wnosi
\newline
- jak niewelowane są opóźnienia
\newline
- która z nich daje najlepsze wyniki
\newline
- opóźnienia i tak są spowodowane przez użycie świateł (światło pomarańczowe)
\newline
\newline
Opis koordynacji ruchu przy użyciu algorytmu REAL TIME QUEUE LENGTH ESTIMATION: THE APTTC ALGORITHM na podstawie artykułu ~\cite{athmaraman2005adaptive}:
  \newline
  - Sterowanie adaptacyjne
  \newline
  - Statystyczna optymalizacja
  \newline
  - Estymacja na podstawia długości dynamicznej kolejki
  \newline
\newline
Porównanie powyższego rozwiązania z synchroniczną zmianą świateł.
\newline
- jakie zalety i wady dają oba te rozwiązania
\newline
- którego rozwiązania używać a jakiej sytuacji
\newline
\newline
Opis kierowania ruchem drogowym na jednym skrzyżowaniu. AUtorzy artykułu \cite{de1998optimal} dla ułatwienia pomijają światło pomarańczowe:
  \newline
  - Optymalizacja świateł za pomocą dwóch podanych funkcji polegających na optymalicazji kolejek na czterech pasach
  \newline
  - Rozwiązanie pomija element bezpieczeństwa jakim jest światło pomarańczowe (w moim rozwiązaniu można je zapewnić)
\newline
\newline
Kontrola ruchu drogowego za pomocą świateł. Autorzy atykułu \cite{lammer2008self} zaproponowali zdecentralizowany algorytm bazujący na krótkich prognozach ruchu:
  \newline
  - Optymalizacja swiatel za pomoca dwoch podanych funkcji polegajacych na optymalizacji kolejek na czterech pasach
  \newline
	- Liczona długość światła zielonego w celu zwolnienia kolejki na pasie
  \newline
  - Limitacja - Zielone światła są zapalone dłużej niż zwykle powinny
  \newline
	- Limitacja - Ruch jest optymalizowany dla sytuacji "średnich", które tak na prawdę nigdy nie występują, przez co optymalizacja nie jest stosowana dla aktualnej sytuacji
  \newline
  \newline
Zarządzanie światłami za pomocą komunikacji pojazdów między sobą \cite{ferreira2010self}
\newline
Wszystkie auta muszą być wyposażone w:
\newline
	- Urządzenia DSRC
\newline
  - Tą sama wersję mapy
\newline
	- GPS z dokładnością do pasa na którym są
\newline
Sieć bezprzewodowa w każdym z samochdów musi być niezawodna
\newline
Rozwiązanie w porównaniu do mojego jest kosztowne, jeżeli chodzi o wyposażenie. Jest ono możliwe także tylko wtedy kiedy wszystkie auta są odpowiednio wyposażone i sprawne.
    
\section{Planowanie ruchu bez sygnalizacji świetlnej}

CEL:
\newline
- przedstawienie algorytmów bez sygnalizacji świetlnej, których można użyć do planowania ruchu
\newline
- porównanie tych rozwiązań między sobą
\newline
- zastosowania tych rozwiązań, przykłady
\newline
\newline
Znajdowanie ścieżki za pomocą algorytmu Dijkstr'y na podstawie artykułu ~\cite{shaikh2013agv} 
\newline
- opisanie sposobu zastosowanie dijkstry w planowaniu ruchu
\newline
- ograniczenia algorytmu
\newline
\newline
Unikanie kolizji przy użyciu algorytmu Dijkstry' na podstawie artykułu ~\cite{shaikh2013agv} 
\newline
\newline
Autonomiczne podejśćie - użycie Automated Guided Vehicels.
\newline
- zmiana podejścia - zarządzanie pojedyńczym samochodem zamiast algorytm dla wszystkich
\newline
- wady i zalety takiego rozwiązania
\newline
\newline
Autorzy artykułu ~\cite{huang2013improved} przedstawiają ulepszoną wersję algorytmu Dijkstry w celu znalezienia najkrótszej ścieżki.
\newline
- Algorytm Dijkstry ma słabą wydajność, dlatego postanowiono go zmodyfikować. Według badań przeprowadzonych przez autorów algorytm jest ~42%-76% szybszy
\newline
- Autorzy wspominają o algorytmie A*, który jest szybszy od algorytmu Dijkstry ale może skończyć w nieskończonej pętli.
\newline
- W prowadzonych przeze mnie badaniach algorytm A* ani raz nie został wprowadzony w nieskończoną pętlę oraz jest szybszy od algorytmu Dijkstry w szukaniu ścieżki.
\newline
\newline
Planowanie ruchu oraz unikanie kolizji przy użyciu AGV na podstawie artykułu ~\cite{szczerba2000robust}
\newline
- opis zastosowanego rozwiązania
\newline
- opis sposobu unikania kolizji
\newline
\newline
Omówienie przykładu z użyciem AGV na podstawie artykułu ~\cite{szczerba2000robust}
\newline
\newline
Omówienie i porównanie agorytmów stosowanych do planowania drogi na podstawie artykułu ~\cite{delling2009engineering}:
  \newline
  - Dijkstra's Algorithm
  \newline
  - Priority Queues
  \newline
  - Bidirectional Search
  \newline
  - A*
  \newline
\newline
\newline
Autorzy artykułu ~\cite{gazis1997optimal} przedstawiają system planujący ruch pojazdów połączonych za pomocą bezprzewodowej komunikacji. System dostarcza pojazdą optymalną trasę biorąc pod uwagę aktualny ruch na drogach.
\newline
- Użycie algorytmu Dijkstry w celu znalezienia najkrótszej ścieżki
\newline
- Celem patentu znalezienie drogi do celu w najkrótszym czasie poprzez omijanie zakorkowanych dróg
\newline
- Jest to jedynie znajdowanie najszbszej drogi do celu poprzez komunikacje bezprzewodową. Nie jest to dokładne zaplanowanie ruchu pojazdu wraz z unikaniem kolizji
\newline
\newline
W Artykule ~\cite{broxmeyer1994vehicle} autorzy przedstawili system do controli ruchu i unikania kolizji na autostradach.
\newline
- Pod uwage wzięte zostały: zmiany pasów, unikanie kolizji, kontrolowanie trasy pojazdu
\newline
- Unikanie kolizji jest zapewniane poprzez komunikacje przez transmitery radiowe oraz radioodbiorniki zamontowane w każdym z pojazdów
\newline
- System planuje trasy oraz zmiany pasów na autostradach przy czym zapewnione jest bezpieczeństwo kolizji. Rozwiązania nie można jednak zaaplikować do skrzyżowań dróg
\newline
\newline
Autorzy artykułu ~\cite{kanoh2007dynamic} przedstawiają system dynamiczny system planowania ruchu do nawigacji samochodów przy użyciu 'virus genetic algorithms'.
\newline
- Zaproponowany został 'Genetic Algorithm', który powinien dawać lepsze wyniki w porównaniu do algorytmu Dijkstry oraz algorytmu A*.
\newline
- Brak unikania kolizji
\newline

\section{Użycie algorytmu A* do planowania ruchu}

CEL:
\newline
- przytoczenie użyć A* do planowania ruchu
\newline
- pokazanie ograniczeń użytych A*
\newline
\newline
Omówienie algorytmu A* na podstawie artykułu ~\cite{dechter1985generalized} 
\newline
- omówienie optymalności
\newline
- omówienie funkcji heurystyki
\newline
\newline
Planowanie ruchu za pomocą zmodyfikowanego algorytmu A* na podstawie artykułu ~\cite{munteanmobile} 
\newline
- opis sposobu wprowadzenia zmodyfikowanego algorytmu A* do koordynowania ruchu na skrzyżowaniach
\newline
- wady, zalety i ogarniczenia
\newline
\newline
Planowanie ruchu za pomocą wielostanowego algorytmu A* oraz Wavefront na podstawie artykułu ~\cite{wojnicki2015robust} 
\newline
- omówienie zastosowania tych dwóch algorytmów do planowania ruchu
\newline
\newline
Omówienie przykładu użycia Multi-Entity A* na podstawie artykułu ~\cite{wojnicki2015robust} 
\newline
- przytoczenie przykładu
\newline
- omówienie wad, zalet i ogarniczeń
\newline
\newline
Omówienie złożoności Multi-Entity A* na powyższym przykładzie.
\newline
- złożoność jest zależna od liczby stanów w A*
\newline
- obliczenie złożoności i pokazania jak ona rośnie
\newline
\newline

\section{Modyfikacje A*}

CEL:
\newline
- przytoczenie modyfikacji A* (nie ma takiej modyfikacji jak moja)
\newline
\newline
Opis wielostanego A* na podstawie artykułu ~\cite{wojnicki2015robust} 
\newline
\newline
Opis modyfikacji A* w artykule ~\cite{munteanmobile}
\newline
\newline
