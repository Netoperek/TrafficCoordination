\chapter{State of the art} \label{chap:state-of-the-art}

\section{Planowanie ruchu na skrzyżowaniach}

CEL:
\newline
- wskazanie tematyki literatury jaka będzie omawiana
\newline
- jakie algorytmy w literaturze są stosowane do planowania ruchu drogowego
\newline
- algorytmy heurystyczne są preferowane ze względu na złożoność
\newline
\newline
Zależność wydajności od rozmiaru problemu
\newline
- Metody planujące są zawsze zależne od rozmiaru problemu
\newline
- Duża złożoność powoduje stosowanie algorytmów heurystycznych
\newline
- Pokazanie złożoności na przykładowych algorytmach
\newline
\newline
Szybkość zmian w środowisku uniemożliwia użycia algorytmu zajmującego dużo czasu
\newline
- Duża złożoność problemu powoduje, że jeżeli zajdzie zmiana w środowisku, algorytm musi natychmiast wyliczyć poprawny plan
\newline
- Algorytmy heurystyczne są wstanie działać w środowisku, gdzie zmiany zachodzą szybko
\newline
\newline
Rozważana klasa problemów
\newline
- Omówienie klasy problemów planowania ruchu na skrzyżowań
\newline
- Omówienie modelów środowisk i wiążących się z nimi złożoności
\newline
\newline

\section{Planowanie ruchu przy użyciu świateł drogowych}

CEL:
\newline
- omówienie literatury rozwiązującej ruch drogowy przy użyciu świateł drogowych
\newline
- światła drogowe wprowadzają opóźnienie spowodowane światłem pomarańczowym
\newline
- omówienie optymalizacji świateł drogowych, które mimo wszystko powodują opóźnienia
\newline
\newline
Opis optymalizacji wprowadzonych do zarządzania sygnalizacją świetlną na podstawie artykułu ~\cite{brockfeld2001optimizing}:
  \newline
  - Synchronized Traffic Lights
  \newline
  - Green Wave
  \newline
  - Random Offset
\newline
\newline
Porównanie powyższych optymalizacji na podstawie artykułu ~\cite{brockfeld2001optimizing}
\newline
- która optymalizacja co wnosi
\newline
- jak niewelowane są opóźnienia
\newline
- która z nich daje najlepsze wyniki
\newline
- opóźnienia i tak są spowodowane przez użycie świateł (światło pomarańczowe)
\newline
\newline
Opis koordynacji ruchu przy użyciu algorytmu REAL TIME QUEUE LENGTH ESTIMATION: THE APTTC ALGORITHM na podstawie artykułu ~\cite{athmaraman2005adaptive}:
  \newline
  - Sterowanie adaptacyjne
  \newline
  - Statystyczna optymalizacja
  \newline
  - Estymacja na podstawia długości dynamicznej kolejki
  \newline
\newline
Porównanie powyższego rozwiązania z synchroniczną zmianą świateł.
\newline
- jakie zalety i wady dają oba te rozwiązania
\newline
- którego rozwiązania używać a jakiej sytuacji
\newline
\newline

\section{Planowanie ruchu bez sygnalizacji świetlnej}

CEL:
\newline
- przedstawienie algorytmów bez sygnalizacji świetlnej, których można użyć do planowania ruchu
\newline
- porównanie tych rozwiązań między sobą
\newline
- zastosowania tych rozwiązań, przykłady
\newline
\newline
Znajdowanie ścieżki za pomocą algorytmu Dijkstr'y na podstawie artykułu ~\cite{shaikh2013agv} 
\newline
- opisanie sposobu zastosowanie dijkstry w planowaniu ruchu
\newline
- ograniczenia algorytmu
\newline
\newline
Unikanie kolizji przy użyciu algorytmu Dijkstry' na podstawie artykułu ~\cite{shaikh2013agv} 
\newline
\newline
Autonomiczne podejśćie - użycie Automated Guided Vehicels.
\newline
- zmiana podejścia - zarządzanie pojedyńczym samochodem zamiast algorytm dla wszystkich
\newline
- wady i zalety takiego rozwiązania
\newline
\newline
Planowanie ruchu oraz unikanie kolizji przy użyciu AGV na podstawie artykułu ~\cite{szczerba2000robust}
\newline
- opis zastosowanego rozwiązania
\newline
- opis sposobu unikania kolizji
\newline
\newline
Omówienie przykładu z użyciem AGV na podstawie artykułu ~\cite{szczerba2000robust}
\newline
\newline
Omówienie i porównanie agorytmów stosowanych do planowania drogi na podstawie artykułu ~\cite{delling2009engineering}:
  \newline
  - Dijkstra's Algorithm
  \newline
  - Priority Queues
  \newline
  - Bidirectional Search
  \newline
  - A*
  \newline
\newline
\newline

\section{Użycie algorytmu A* do planowania ruchu}

CEL:
\newline
- przytoczenie użyć A* do planowania ruchu
\newline
- pokazanie ograniczeń użytych A*
\newline
\newline
Omówienie algorytmu A* na podstawie artykułu ~\cite{dechter1985generalized} 
\newline
- omówienie optymalności
\newline
- omówienie funkcji heurystyki
\newline
\newline
Planowanie ruchu za pomocą zmodyfikowanego algorytmu A* na podstawie artykułu ~\cite{munteanmobile} 
\newline
- opis sposobu wprowadzenia zmodyfikowanego algorytmu A* do koordynowania ruchu na skrzyżowaniach
\newline
- wady, zalety i ogarniczenia
\newline
\newline
Planowanie ruchu za pomocą wielostanowego algorytmu A* oraz Wavefront na podstawie artykułu ~\cite{wojnicki2015robust} 
\newline
- omówienie zastosowania tych dwóch algorytmów do planowania ruchu
\newline
\newline
Omówienie przykładu użycia Multi-Entity A* na podstawie artykułu ~\cite{wojnicki2015robust} 
\newline
- przytoczenie przykładu
\newline
- omówienie wad, zalet i ogarniczeń
\newline
\newline
Omówienie złożoności Multi-Entity A* na powyższym przykładzie.
\newline
- złożoność jest zależna od liczby stanów w A*
\newline
- obliczenie złożoności i pokazania jak ona rośnie
\newline
\newline

\section{Modyfikacje A*}

CEL:
\newline
- przytoczenie modyfikacji A* (nie ma takiej modyfikacji jak moja)
\newline
\newline
Opis wielostanego A* na podstawie artykułu ~\cite{wojnicki2015robust} 
\newline
\newline
Opis modyfikacji A* w artykule ~\cite{munteanmobile}
\newline
\newline
