\chapter{State of the art} \label{chap:state-of-the-art}

\section{Planowanie ruchu na skrzyżowaniach}

Zależność wydajności od rozmiaru problemu
\newline
\newline
Szybkość zmian w środowisku uniemożliwia użycia algorytmu zajmującego dużo czasu
\newline
\newline
Rozważana klasa problemów
\newline
\newline
Porównanie metody deterministycznej i heurystycznej
\newline
\newline
Konieczność użycia metody heurystycznej z powodu dużej złożoności

\section{Planowanie ruchu przy użyciu świateł drogowych}

Opis dotychczasowych modeli ruchu przy użyciu świateł drogowych na podstawie artykułu [1]
\newline
\newline
Opis optymalizacji wprowadzonych do zarządzania sygnalizacją świetlną na podstawie artykułu [1]:
  \newline
  - Synchronized Traffic Lights
  \newline
  - Green Wave
  \newline
  - Random Offset
  \newline
\newline
\newline
Porównanie powyższych optymalizacji na podstawie artykułu [1]
\newline
\newline
Opis koordynacji ruchu przy użyciu algorytmu REAL TIME QUEUE LENGTH ESTIMATION: THE APTTC ALGORITHM na podstawie artykułu [7]:
  \newline
  - Sterowanie adaptacyjne
  \newline
  - Statystyczna optymalizacja
  \newline
  - Estymacja na podstawia długości dynamicznej kolejki
  \newline

Porównanie powyższego rozwiązania z synchroniczną zmianą świateł.
\newline
\newline

\section{Planowanie ruchu bez sygnalizacji świetlnej}

Znajdowanie ścieżki za pomocą algorytmu Dijkstr'y na podstawie artykułu [2]
\newline
\newline
Unikanie kolizji przy użyciu algorytmu Dijkstry' na podstawie artykułu [2]
\newline
\newline
Planowanie ruchu za pomocą zmodyfikowanego algorytmu A* na podstawie artykułu [3]
\newline
\newline
Planowanie ruchu za pomocą wielostanowego algorytmu A* oraz wielostanowego algorytmu Wavefront na podstawie artykułu [4]
\newline
\newline
Autonomiczne podejśćie - użycie Automated Guided Vehicels.
\newline
\newline
Planowanie ruchu oraz unikanie kolizji przy użyciu AGV na podstawie artykułu [8]
\newline
\newline
Omówienie przykładu z użyciem AGV na podstawie artykułu [8]
\newline
\newline
Omówienie agorytmów stosowanych do planowania drogi na podstawie artykułu [5]:
  \newline
  - Dijkstra's Algorithm
  \newline
  - Priority Queues
  \newline
  - Bidirectional Search
  \newline
  - A*
  \newline
\newline
\newline

\section{Użycie algorytmu A* do planowania ruchu}

Omówienie użycia A* do rozwiązywania problemu znajdowania drogi na podstawie artykułu [6]
\newline
\newline
Omówienie algorytmu A* na podstawie artykułu [6]
\newline
\newline
Omówienie przykładu użycia Multi-Entity A* na podstawie artykułu [4]
\newline
\newline
Omówienie złożoności Multi-Entity A* na powyższym.
\newline
\newline

\section{Modyfikacje A*}

Opis wielostanego A* na podstawie artykułu [4]
\newline
\newline
Opis modyfikacji i użycia A* w artykule [3]
\newline
\newline
