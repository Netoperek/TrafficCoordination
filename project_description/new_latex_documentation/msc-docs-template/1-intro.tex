\chapter{Wstęp}

\section{Dziedzina Problemu}

W pracy rozwiązywany jest problem planowania ruchu na skrzyżowaniach.
\newline
\newline
W rozwiązaniu wykorzystany jest algorytm A*, który najczęściej służy do wyszukiwania najkrótszej ścieżki w grafie.
\newline
\newline
Poruszony jest także problem unikania kolizji wraz z użyciem algorytmu A*.
\newline
\newline
Omówiona jest trudność w planowaniu ruchu globalnego jakim jest złożoność, która róśnie wraz z rozmiarem środowiska.
\newline
\newline
Planowanie ruchu na skrzyżowaniach wiaże się z czestymi zmianami środowiska. W związku z tym system planujący dokładnie ruch powinien być wstanie szybko reagować na pomyłki.

\section{Specyfikacja problemu}

Problemem przedstawionym w pracy jest szybkie i bezpieczne zaplanowanie ruchu na skrzyżowaniach. Mając określony stan początkowy będącym położeniem wszystkich pojazdów na skrzyżowaniu, system zwraca dokładny plan, który prowadzi do szybkiego i bezpiecznego opuszczenia przez samochody skrzyżowań.
\newline
\newline
Problem polega na znalezieniu kolejnych stanów prowadzących do optymalnego rozwiązania. Pojedyńczy stan oraz jego zmiany wprowadzają duża liczbę możliwości. Dla przykładu na drodze reprezentowanej w sposób dyskretny o rozmiarze 24 oraz pojeździe znajdującej się w pozycji 12 - pojazd może zwolnić, przyspieszyć, lub przyspieszenia nie zmieniać.
\newline
\newline
Kolejną trudnością w problemie, jest unikanie kolizji. Dla wszystkich możliwość algorytm planowania, powinien usunąć wszystkie możliwości kolizji na jednym pasie oraz przy przekraczaniu przez skrzyżowanie.

\section{Struktura pracy}
TODO
