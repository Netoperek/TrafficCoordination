\chapter{Wstęp}

\section{Dziedzina Problemu}

W pracy rozwiązywany jest problem planowania ruchu na skrzyżowaniach.
\newline
\newline
W rozwiązaniu wykorzystany jest algorytm A*, który najczęściej służy do wyszukiwania najkrótszej ścieżki w grafie.
\newline
\newline
Poruszony jest także problem unikania kolizji wraz z użyciem algorytmu A*.
\newline
\newline
Omówiona jest trudność w planowaniu ruchu globalnego jakim jest złożoność, która róśnie wraz z rozmiarem środowiska.
\newline
\newline
Planowanie ruchu na skrzyżowaniach wiaże się z czestymi zmianami środowiska. W związku z tym system planujący dokładnie ruch powinien być wstanie szybko reagować na pomyłki.

\section{Specyfikacja problemu}

Problemem przedstawionym w pracy jest szybkie i bezpieczne zaplanowanie ruchu na skrzyżowaniach. 
\newline
\newline
Mając określony stan początkowy będącym położeniem wszystkich pojazdów na skrzyżowaniu, system zwraca dokładny plan, który prowadzi do szybkiego i bezpiecznego opuszczenia przez samochody skrzyżowań.
\newline
\newline
Problem polega na znalezieniu kolejnych stanów prowadzących do optymalnego rozwiązania. Pojedyńczy stan oraz jego zmiany wprowadzają dużą liczbę możliwości.
\newline
\newline
Dla przykładu pojazd musi mieć określoną pozycję na drodze na której się znajduje oraz prędkość. W kolejnych stanach pojazd może zmienić położenie, przyspieszyć, zwolnić lub prędkości nie zmieniać.
\newline
\newline
Kolejną trudnością w problemie, jest unikanie kolizji. Dla wszystkich możliwość algorytm planowania, powinien usunąć wszystkie możliwości kolizji na jednym pasie oraz przy przekraczaniu przez skrzyżowanie.

\section{Cel}

Celem pracy jest opracowanie metody planującej ruch drogowy na skrzyżowaniach dróg wielopasmowych przy użyciu zmodyfikowanego algorytmu A*. 
\newline
\newline
Wynikiem rozwiązania jest szczegółowy opis, który dokładnie planuje ruch każdego z samochodów w celu najszybszego i bezpiecznego opuszczenia skrzyżowania.
\newline
\newline
System zapewnia unikanie kolizji oraz optymalny dojazd do celu pojazdów na zadanej sieci skrzyżowań.

\section{Struktura pracy}
TODO
