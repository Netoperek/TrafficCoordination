
\chapter{State of the art}

\section{Planowanie ruchu na skrzyżowaniach}

Zależność wydajności od rozmiaru problemu

Szybkość zmian w środowisku uniemożliwia użycia algorytmu zajmującego dużo czasu

Rozważana klasa problemów

Porównanie metody deterministycznej i heurystycznej

Konieczność użycia metody heurystycznej z powodu dużej złożoności
\bt

\section{Planowanie ruchu przy użyciu świateł drogowych}

Opis dotychczasowych modeli ruchu przy użyciu świateł drogowych na podstawie artykułu [1]

Opis optymalizacji wprowadzonych do zarządzania sygnalizacją świetlną na podstawie artykułu [1]:
  - Synchronized Traffic Lights
  - Green Wave
  - Random Offset

Porównanie powyższych optymalizacji na podstawie artykułu [1]

Opis koordynacji ruchu przy użyciu algorytmu REAL TIME QUEUE LENGTH ESTIMATION: THE APTTC ALGORITHM na podstawie artykułu [7]:
  - Sterowanie adaptacyjne
  - Statystyczna optymalizacja
  - Estymacja na podstawia długości dynamicznej kolejki

Porównanie powyższego rozwiązania z synchroniczną zmianą świateł.

\section{Planowanie ruchu bez sygnalizacji świetlnej}

\bt

Znajdowanie ścieżki za pomocą algorytmu Dijkstr'y na podstawie artykułu [2]

Unikanie kolizji przy użyciu algorytmu Dijkstry' na podstawie artykułu [2]

Planowanie ruchu za pomocą zmodyfikowanego algorytmu A* na podstawie artykułu [3]

Planowanie ruchu za pomocą wielostanowego algorytmu A* oraz wielostanowego algorytmu Wavefront na podstawie artykułu [4]

Autonomiczne podejśćie - użycie Automated Guided Vehicels.

Planowanie ruchu oraz unikanie kolizji przy użyciu AGV na podstawie artykułu [8]

Omówienie przykładu z użyciem AGV na podstawie artykułu [8]

Omówienie agorytmów stosowanych do planowania drogi na podstawie artykułu [5]:
- Dijkstra's Algorithm
- Priority Queues
- Bidirectional Search
- A*

\section{State of the art}

\subsection{Technology 1}

A lot of bibliography citations here...

\bt

\subsection{Technology 2}

\bt

\section{Main thesis of this work}

The main thesis of this work may be expressed as follows:

\medskip

\textit{
	\bt
}

